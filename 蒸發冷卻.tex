\documentclass[12]{article}
\usepackage{enumerate} %列举宏包

\usepackage{longtable}%长表格

\usepackage{multirow} %合并多行单元格的宏包
\usepackage{longtable} %不宽但很长的表格可以用longtable宏包来进行分页显示
\usepackage{array} %一般用于数学公式中对数组或矩阵的排版
\usepackage{makecell}% makecell命令对表格单元格中的数据进行一些变换的控制。我们可以使用 \ 命令进行换行,也可以使用p{(宽度)}选项控制列表的宽度
\usepackage{threeparttable} %制作三线表格
\usepackage{booktabs}%s三线表格中的上中下直线线型设置宏包,在这个包中水平线被教程\toprule、midrule、buttomrule。
%表头文字格式的详细设置
\renewcommand\theadset{\renewcommand\arraystretch{0.85}%
\setlength\extrarowheight{0pt}}%行距
\renewcommand\theadfont{\small}%字体
\renewcommand\theadalign{rt}%行列对齐
\renewcommand\theadgape{\Gape[0.3cm][2mm]}%上下垂直距离

\usepackage{tocbibind}
 
\usepackage{indentfirst}%设置首行缩进1.5字符
\setlength{\parindent}{1.5em}%设置首行缩进1.5字符
\usepackage{caption}%规范了图片和表格的命名方式
\captionsetup[figure]{labelfont={bf},labelformat={default},labelsep=period,name={Fig.}} %图片命名时,默认为"Figure 1",但我查看很多期刊,都采用"Fig. 1",即"简写+加粗"的形式,第1行代码实现的就是这种转化。
%\captionsetup[table]{font={small}, labelsep=none, labelfont={bf}, labelsep=newline,singlelinecheck=off} %表格标题命名时,默认是居中且仅有一行。但大家仔细查看英文论文时会发现,表格(Table1)标题(表格题目)都是居左的,有两行。故第2行代码实现的就是该功能。
\linespread{1}%设置行间距
\renewcommand{\baselinestretch}{1.5} \normalsize%这种行间距的写法是针对全文而言的,若想2倍行间距,把数值1.5改成2即可;
%\usepackage[backref]{hyperref}%让引用位置和参考文献位置连接互通: (无法使用
\usepackage{cite}%引用
\usepackage{geometry}
%\usepackage{fontspec}%设置字体 (与粗体斜体设置冲突,故取消
 %\setmainfont{Times New Roman}%using Times New Roman字体
\usepackage{fancyhdr}%页眉与页脚
\pagestyle{fancy}%页眉与页脚
\fancyhf{}%清除原页眉页脚样式
\fancyhead[L]{ \MakeUppercase{Evaporative Cooling OF TRAPPED ATOMS} }%左页眉
%\fancyhead[R]{ Verification of special relativity $\&$ $\beta$ spectroscopy}%右页眉
\usepackage{graphicx} %插入图片的宏包
\usepackage{hyperref}
\usepackage{float} %设置图片浮动位置的宏
\usepackage{subfigure} %插入多图时用子图显示的宏包
\geometry{a4paper ,scale=0.85}%设置纸张大小与行宽
\makeatletter%罗马字母的使用 
\newcommand{\rmnum}[1]{\romannumeral #1}
\newcommand{\Rmnum}[1]{\expandafter\@slowromancap\romannumeral #1@}
\makeatother%小写罗马数字 : \rmnum{数字} 大写罗马数字 : \Rmnum{数字}
\title{Evaporative Cooling }\author{LHY}
\fancyhead[C]{\subsectionmark}
\fancyfoot[C]{\sectionmark}
\begin{document}
\maketitle%report从这里开始
\setcounter{page}{0}%让封面不显示页数
\thispagestyle{empty}%让封面不显示页数
%\renewcommand{\abstractname}{\Large \textbf% 粗体设置
%{Abstract}\\}%让Abstract这个标题变大
\renewcommand{\contentsname}{Summary}
\setcounter{tocdepth}{2}
\tableofcontents
 
\newpage
\section{Theoretical Models for Evaporative Cooling}
\subsection{General Scaling Laws}
First, evaporative cooling happens on an\textbf{\textit{ exponential scale}}: Within a certain time interval (naturally measured in units of collision times or relaxation times), all relevant parameters (number of atoms, temperature, density) change by a certain factor. The characteristic quantities for the evaporation process are therefore logarithmic derivatives such as: $$\alpha = \frac{d(ln\ T)}{d(ln\ N)} = \frac{\dot{T}/T}{\dot{N}/N}$$
  
  
  Or, if evaporative cooling is described as a process with finite steps, we have: $$\alpha = \frac{ln\ (T'/T)}{ln\ (N'/N)},\ T'=T + \Delta T,\ N'=N+\Delta N$$
  
  If $\alpha$ is constant during the evaporation process, the temperature drops with function: $$\frac{T(t)}{T(0)}=(\frac{N(t)}{N(0)})^{\alpha} $$
  
  In a power law potential in $d$ dimensions, $U(r)\propto r^{d/\delta}$,\textbf{\textit{ all relevant quantities}} scale as $[N(t)/N(0)]^x $ during evaporative cooling, where $x$ depends only on $\delta$ and $\alpha$. $\delta$ is defined in such a way that the volume scales as $T^\delta$. All other quantities are products of powers of temperature, number, and volume, and their scaling is listed in Table 1.

\begin{table}[htbp]
  \centering
  \caption{Scaling Laws For Evaporative Cooling In \\ A $d$ Dimension Potential $U(r)\propto r^{d/ \delta}$ }
  \begin{tabular}{|l|l|c|}
    \hline
    Quantity & Exponent $^a, x$ \\
   \hline Number of atoms, $N$ & 1 \\
Temperature, $T$ & $\alpha$ \\
Volume, $V$ & $\delta \alpha$ \\
Density, $n$ & $1-\delta \alpha$ \\
Phase-space density, $D$ & $1-\alpha(\delta+3 / 2)$ \\
Elastic collision rate, $n \sigma v$ & $1-\alpha(\delta-1 / 2)$ \\
\hline
  \end{tabular}
\end{table}

Phases space density $D$ is defined as $n\lambda^3_{dB}$ with the thermal de Broglie wavelength $\lambda_{dB} = \sqrt{\frac{2\pi \hbar^2}{mkT}}$, for an atom with mass $m$. 

The key parameter of the whole cooling process is $\alpha$ , which \textbf{\textit{expresses the temperature decrease per particle los}}t. Technically, evaporation is controlled by limiting the depth of the potential to $\eta kT$. \textbf{\textit{The average energy of the escaping atoms}} is $(\eta + \kappa)kT$, where $\kappa$ is a small number usually between 0 and 1, depending both on $\eta$ and on the dimension of the evaporation.  

For large $\eta$, the energy distribution of the trapped atoms is close to a Boltzmann distribution with an average energy of $(\delta+\frac{3}{2})kT$, and there is \textbf{\textit{a simple relation between the average energy of an escaping atom and $\alpha$:}} $$\alpha= \frac{\eta+\kappa}{\delta+\frac{3}{2}}-1 =\frac{(\eta+\kappa-\delta-\frac{3}{2})\ kT}{(\delta+\frac{3}{2})\ kT}$$

This expression has an obvious meaning: $\alpha$ is a dimensionless quantity, charactericing how much more than the average energy $(\delta+\frac{3}{2})\ kT$ is removed by an evaporating atoms. These considerations show that in principle there is no upper bound for $\alpha$ or the efficiency of evaporative cooling. Therefore, efficiency of evaporation, and comparison between different trap geometries, can only be made if the trade-off between efficiency and cooling speed is considered. This is done by specifying loss mechanisms that are unavoidable in practice.
\subsection{The Speed of Evaporation Loss Processes}
We now want to extend the discussion and consider the speed of evaporation, introduce time as a parameter. The situation we have in mind is forced evaporation at a constant $\eta$ parameter; the threshold for evaporation is lowered in proportion to the decreasing temperature. Constant $\eta$ ensures that the energy distribution is only rescaled during the cooling and does not change its shape. This assumption is reasonably well fulfilled in experiments.

\textbf{\textit{The principle of detailed balance}} can be used in kinetic systems which are decomposed into elementary processes (collisions, or steps, or elementary reactions). It states that at equilibrium, each elementary process is in equilibrium with its reverse process. $Rate(A\rightarrow B)=Rate(B\rightarrow A)$

We consider particles at density $n_0$ in a box potential, and we assume that $\eta$ is large. The rate of evaporating atoms can then be obtained as follows: In an untruncated Maxwell-Boltzmann distribution, almost every collision involving an atom in the high energy tail removes the atom from the high energy tail. By detailed balance, elastic collisions produce atoms with energy larger than $\eta kT$ at a rate that is simply the number of atoms with energy larger than $\eta kT$ divided by their collision time. For a large value of $\eta$,the rate of evaporation in a truncated Boltzmann distribution is identical to the production rate of atoms with energy larger than $\eta kT$ in the untruncated distribution. 

The velocity of atoms with energy $\eta kT$ is $\sqrt{\frac{2\eta kT}{m}}=\sqrt{\eta \pi }\cdot \frac{\bar v}{2}$, where $\bar{v}$ denotes the average thermal velocity. For large $\eta$, the fraction of atoms with energy larger than $\eta kT$ approaches $2e^{-\eta} \sqrt{\frac{\eta}{\pi}}$. The rate of evaporating atoms is thus: $$ \dot{N}=-N n_0 \sigma \bar{v} \eta e^{-\eta}=\frac{-N}{\tau_{\mathrm{ev}}} $$

Where we have introduced the elastic collision cross section $\sigma$ and the time constant for evaporation $\tau_{ev}$.   It's ratio to the elastic collision time $\tau_{el} $ is expressed by $\lambda=\frac{\tau_{ev}}{\tau_{el}}$ Using $\frac{1}{\tau_{el}}=n_0 \sigma \bar{v} \sqrt{2}$, where $\bar{v}\sqrt{2}$ is the average relative velocity between two atoms, we obtain in the limit of large $\eta$, $$\lambda=\frac{\sqrt{2}e^\eta}{\eta}$$
\subsubsection{Runaway Evaporation}
For alkali atoms, where the dominant loss mechanism is background gas collisions, an important criterion for sustained evaporation is to maintain or increase the elastic collision rate $n \sigma v$. It follows that \textbf{\textit{the elastic collision rate varies}} as: $$ \frac{d(n \sigma v)}{d t} / n \sigma v=\frac{1}{\tau_{\mathrm{el}}}\left(\frac{\alpha(\delta-1 / 2)-1}{\lambda}-\frac{1}{R}\right)$$ 

In the temperature range of interest, $\sigma$ is the s-wave cross section and is independent of temperature. $R=\frac{\tau_{loss}}{\tau_{el}}$ is the number of elastic collisions per trapping time (also called the ratio of good to bad collisions) where $\tau_{loss} $ is the time constant for trap loss due to background gas collisions. Evaporation at constant or increasing collision rate ("runaway evaporation") requires: $$R \geq R_{\min }=\frac{\lambda}{\alpha(\delta-1 / 2)-1}$$

In the absence of any loss process $( R = \infty)$, the minimum $\eta$ for runaway evaporation is determined by: $$\alpha>\frac{1}{\delta-1 / 2}$$

\textbf{\textit{The increase of phase space density}} with time is given by:$$
\beta=100 \tau_{\mathrm{el}} \frac{d}{d t}\left(\log _{10} D\right)=\frac{100}{\ln 10}\left(\frac{\alpha(\delta+3 / 2)-1}{\lambda}-\frac{1}{R}\right)$$

\subsubsection{Maximizing Phase-Space Density}
$\alpha$ describes only the change in temperature. If we regard evaporation in a linear potential as analogous to evaporation in a harmonic potential with continuous adiabatic compression, we realize that a does not provide the most meaningful comparison, because adiabatic compression trades in temperature against density. We therefore now focus on phase-space density $D$, which is invariant with respect to adiabatic changes of the potential.

\textit{\textbf{The relative increase in phase space density with decreasing number N}}:$$
\gamma=-\frac{d(\ln D)}{d(\ln N)}=\frac{\alpha(\delta+3 / 2)}{1+\lambda / R}-1$$

The most important figure of merit of evaporative cooling in atom traps is to \textbf{\textit{achieve the maximum increase in phase-space density with the smallest loss in the number}} (to reach BEC with the largest number of atoms possible). This would mean that the goal is to achieve the largest value of the global parameter:$$\gamma_{tot}=\frac{\ln \left(D_{final} / D_{initial}\right)}{\ln \left(N_{final} / N_ {initial}\right)}$$

This global parameter is optimized by optimizing $\gamma$ at any moment. To explain this, we first note that the elastic collision rate at any moment can be expressed by the phase-space density and the number of atoms:$$n \sigma v \propto D^{(\delta-1 / 2) /(\delta+3 / 2)} N^{2 /(\delta+3 / 2)}$$

Then we divide the increase in phase-space density in many small steps of size $\Delta D$. The efficiency y of the second step is optimized by maximizing $R$. According to last equation, this is achieved by maximizing the number of atoms left after the first step. As a result, following an evaporation path that maximizes $\gamma$ at any given point maximizes $\gamma_{tot}$ and, thus, the number of atoms left at the final phase-space density.

The same conclusion applies for cooling of atomic hydrogen, where the dominant loss process is inelastic binary collision. $R$ is then independent of density and proportional to $\sqrt{T}$, resulting in:$$R \propto D^{-1 /[2(\delta+3 / 2)]} N^{1 /[2(\delta+3 / 2)]}$$

There is \textbf{\textit{one major difference}} between the loss due to background gas collisions and that due to inelastic collisions. Inelastic collisions heat up the sample because they happen most frequently in regions of high density, where the potential energy is small. As a result, atoms lost due to \textbf{\textit{inelastic collisions}} carry away less than their share in total energy, thus increasing the average energy of the trapped atoms. This effect decreases the effective $\alpha$, which expresses the average energy of a lost particle. Actually, $\alpha$ even changes sign as a function of temperature when the heating due to \textbf{\textit{dipolar relaxation}} dominates over the evaporative cooling. The criterion $\alpha = 0$ determines the lowest temperature that can be achieved in evaporative cooling due to relaxation heating, whereas a model considering only background gas collisions allows cooling to arbitrarily small temperatures.
\subsubsection{Strategies for Evaporation Cooling}
It might appear that the two previous subsections on phase-space density increase and runaway evaporation describe different strategies or even conflicting goals. We now want to discuss qualitatively how they depend on each other and emphasize \textbf{\textit{the major difference between evaporative cooling in alkalis and in atomic hydrogen}}.

\textbf{\textit{The ultimate goal is the achievement of high phase-space densities.}} As was pointed out, maximizing $\gamma$ is the optimum strategy for this goal. However, for small $R$, $\gamma$ becomes negative; no increase in phase-space density is possible. Therefore, a phase-space density increase can only be sustained as long as $R$ is larger than a critical value. 

For loss due to \textbf{\textit{background gas collisions}} (the dominant limitation for alkalis), $R$ changes according to:$$
-\frac{d(\ln R)}{d(\ln N)}=\frac{\alpha(\delta-1 / 2)}{1+\lambda / R}-1$$ 

$R$ varies exponentially with $\frac{1}{N}$, with the exponent given by the right hand side of equation. If the exponent is negative at some point during the cooling,  $R$ decreases, resulting in an even more negative exponent later on.  The consequence is an accelerated decrease of $R$ and an accelerated reduction of the efficiency parameter $\gamma$, until eventually $\gamma$ reaches zero and no further increase in phase-space density is possible. Work in \textbf{\textit{alkali systems}} concentrated on realizing an initial situation with a sufficiently large $R$ , so that $R$ increased during evaporation. In principle, cooling with a (slowly) decreasing $R$ is possible, as long as the desired phase-space density is reached before $\gamma$ becomes zero. However, when the threshold of \textbf{\textit{runaway evaporation}} is reached, $R$ stays constant or increases. This guarantees \textbf{\textit{both fast and efficient cooling to very high phase-space densities}}. The phase-space density increase is only limited by the onset of other loss processes such as \textbf{\textit{three-body recombination and dipolar relaxation}}, which inevitably reduce $R$ and throttle the cooling process. 

In \textbf{\textit{atomic hydrogen}}, the situation is quite different. All experiments in atomic hydrogen were done in a cryogenic environment where \textbf{\textit{background gas collisions were negligible}}, and the dominant loss mechanism was \textbf{\textit{inelastic binary collisions}}. $R$ is \textbf{\textit{independent of density and decreases proportionally to $\sqrt{T}$}}. Runaway or increasing collision rates speed up the cooling process, but in contrast to alkali atoms, they do not increase $R$ nor improve the efficiency of evaporative cooling. In all experiments, the initial temperature, and therefore $R$, was large enough to allow efficient cooling in the beginning. However, during the cooling process the efficiency decreased, and the highest phase-space densities were reached when either $\gamma$ became small or the number of atoms in the sample reached the detection limit.

 \subsection{Models}
 \subsubsection{Amsterdam}
 The Amsterdam group modeled evaporative cooling within a kinetic theory involving a numerical solution of the Boltzmann equation. These results confirmed that \textbf{\textit{the energy distribution can be very well approximated by a truncated Boltzmann distribution}}.  Subsequently, a closed set of differential equations for evaporative cooling was derived under the assumption of a truncated Boltzmann distribution, that all particles with energy larger than $\eta kT$ leave the trap.       
  \subsubsection{Davis and Coworkers (MIT)}
  It approximated the evaporation process as a discrete series of truncation and rethermalication processes, and it arrived at simple analytical results. These were used to discuss the threshold for runaway evaporation for different potentials.     
\subsubsection{Doyle and Coworkers (MIT)}
John Doyle and collaborators derived a set of coupled differential equations that described various cooling, heating, and loss processes and included adiabatic changes of the potential during the cooling process. Furthermore, optimized trajectories in phase space were determined aiming at the highest final phase-space density for a given initial temperature and density combination.
\subsubsection{Monte Carlo Simulations}
Two groups developed an elegant Monte Carlo trajectory technique that allowed an efficient solution of the classical kinetic equation. This technique solves the Boltzmann equation directly without any assumptions such as partial equilibrium or the presence of a truncated Boltzmann distribution.      
\subsection{The Dimension of Evaporation}       
Evaporative cooling relies on the selection of energetic particles that leave the  trap. This selection can be based on the total energy $E$ or on the energy for the motion in a particular direction $E_z$. We call the selection one-dimensional if it is based on $E_z > \eta kT$, two-dimensional for $E_x +E_y > \eta kT$, and three-dimensional for $E > \eta kT$. This distinction is rigorous only for separable potentials. In the context of evaporative cooling the relevant timescale is the time between elastic collisions. Thus, if the ergodic mixing time is longer than the elastic collision time, the potential can be regarded as separable.

In the case of negligible ergodic mixing, the depletion of the high- energy tail of the distribution depends on the selection method. In the opposite case of fast ergodic mixing the distribution is depleted for total energy $E > \eta kT$, independent of the dimensionality of the selection. Since the dynamics of the evaporation are determined by the dimension of the depletion, we will call this \textbf{\textit{the dimension of evaporation.}} This distinction is rigorous only for separable potentials. In the context of evaporative cooling the relevant timescale is the time between elastic collisions. Thus, if the ergodic mixing time is longer than the elastic collision time, the potential can be regarded as separable. 

In the case of negligible ergodic mixing, the depletion of the high- energy tail of the distribution depends on the selection method. In the opposite case of fast ergodic mixing the distribution is depleted for total energy $E > \eta kT$, independent of the dimensionality of the selection. Since the dynamics of the evaporation are determined by the dimension of the depletion, we will call this the dimension of evaporation.        
       
 Most evaporative cooling experiments on hydrogen used evaporation over a saddle point of the potential which is a  $1D$ selection scheme. Radiative evaporation using rf is a three-dimensional selection scheme in a dc magnetic trap. However, in the TOP trap, it is only two-dimensional, and if gravitational forces become essential, it is only one-dimensional.
       
The problem with evaporation in lower dimensions is the dramatic reduction in efficiency for an Ioffe-Pritchard (IP) trap. The reason is that a major fraction of the atoms that have enough total energy to escape collide with other atoms and lose the high excitation energy. Therefore, several “attempts” are necessary before an atom leaves the trap.   
    
As long as there are no loss processes, this affects only the time for evaporation. However, if a time limitation is set by inelastic collision processes and background gas collisions, one- or two-dimensional evaporation strongly decreases the efficiency of evaporative cooling. Since loss processes enter in the ratio $\frac{\lambda}{R}$, an increase in $\lambda$ has an effect similar to a decrease in the ratio of good to bad collisions.
  \subsection{The Number of Collisions to Bose-Einstein Cindensation}
  All the models discussed here show that evaporative cooling is a fast and efficient way to reduce the temperature of trapped atoms and increase the phase-space density. A possible scenario is as follows: harmonic potential, $\eta$ parameter of 6, removal of $0.7\%$ of the atoms per collision time, increase of phase-space density by a factor of $10^6$ in 600 collision times, and a loss in the number by a factor of 100. Reduction of the $\eta$ parameter to 5 gives the same phase-space density increase in half the number of collisions, but with 5 times fewer atoms left.

Since the phase-space density of laser-cooled atoms is typically $10^{-6}$ lower than is required for Bose-Einstein condensation (BEC), one can summarize the potential of evaporative cooling in the following way: About 500 collision times suffice to achieve BEC with $1\%$ of the atoms remaining.

Real experiments have performed somewhat worse, probably due to nonoptimized cooling, lower dimension of the evaporation, and additional loss and heating processes. However, the few experiments done so far have already impressively confirmed the potential of evaporative cooling.
 \subsection{Desirable Extensions of The Models}   
   The theoretical foundations of evaporative cooling are well established. The different models we have discussed describe the salient features of the evaporation process. Extensions in the following directions appear worth- while:
   
   \begin{itemize}
  \item The calculation of optimized trajectories in phase space. There are obviously trade-offs between efficiency and cooling time, and some detailed treatment is desirable.
  \item So far, all models assume that all particles with a certain (1D or 3D) energy will escape. However, all methods of evaporation rely on spatial selection of atoms.
  \item All models discuss evaporation in the dilute regime. Dilute means here collisionally thin, or that the mean free path is longer than the size of the sample.
  \item Most models have treated the atomic motion classically, ignoring quantum statistical effects. For the description of the cooling process, this is justified because evaporative cooling has been used to increase the phase-space density by six orders of magnitude, whereas quantum statistical effects are only pronounced in the vicinity of BEC. 
\end{itemize}
\section{The Role of Collisions for Real Atoms}
\subsection{Elastic And Inelastic Collisions}
\textbf{\textit{Evaporative cooling is driven by elastic collisions.}} High density and the presence of collisions are therefore prerequisites for evaporative cooling. However, this means that inelastic collisions are unavoidable. We have already mentioned background gas collisions and dipolar relaxation as loss processes, and we wish to give a more systematic account of collision processes in this section.

In a magnetic trap, three kinds of inelastic collisions are relevant for evaporative cooling experiments: \textbf{\textit{dipolar relaxation and spin relaxation, which are binary collisions, and three-body recombination (or dimerization).}}

If atoms are trapped in the upper hyperfine state, hyperfine state changing collisions are important, and the loss processes are similar to the situation in a magnetic trap.

\textbf{\textit{Spin relaxation}} has a rather large rate coefficient, typically $10^{-12}\ cm^3/sec$. It involves an exchange of angular momentum between the electron spin and the nuclear spin. This process is usually important only after the initial loading of the trap and leads to a “self-purification” of the sample in the sense that there are only atoms left in a hyperfine state, which does not undergo spin relaxation. This usually happens in the first few seconds after loading a magnetic trap with atomic hydrogen.

Spin relaxation is forbidden for a doubly spin polarized sample due to angular momentum conservation, and also in the upper (weak-field- seeking) sublevel of the lower hyperfine level (as long as kT is small compared with the ground state hyperfine splitting). However, spin-flips can still occur by coupling the spin angular momentum to the orbital angular momentum. This process, called dipolar relaxation, happens with a rate coefficient $G_{dip}$ typically $10^{-15}\ cm^3/sec$, 1000 times smaller than spin relaxation. The rate coefficients for dipolar relaxation are similar for hydrogen and alkalis.

As for all inelastic processes, the rate coefficient $G_{dip}$ is constant for temperatures approaching zero, whereas for elastic collisions the cross section $\sigma$ approaches a constant. This means that in the limit of low temperature and low density (so that three-body recombination can be neglected) the ratio of good to bad collisions is given by $\sqrt{2} \sigma \bar{v} /G_{dip}$.

This ratio reaches unity at a characteristic temperature $T^*$: $$kT^*=\frac{\pi m G^2_{dip}}{16 \sigma^2} $$

This temperature represents a theoretical lower bound to the minimum temperatures attainable by forced evaporation. The minimum temperature in a harmonic trap is about three times larger than $T^*$, .The smallest temperature that can be reached by evaporative cooling with increasing density is approximately 2000 times larger. One might thus regard $1000T^*$, as the practical temperature limit of evaporative cooling.


       
       
       
       
       
       
       
       
       



\begin{center}%居中表示结尾
	\Large***END***
\end{center}


\end{document}